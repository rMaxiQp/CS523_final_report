\section{Case Studies}
Out of the six virtualization layer fault, we chose a couple fo test the effectiveness of the safety guardrail. Due to the scope of the project and our limited funding, we outline a list of cases studies that conducted with our implementation. For each case, we report general statistics, trigger rates, and success rates, followed by a qualitative analysis of observed agent behaviors, blocked commands, and the effectiveness of the system’s guardrails in preventing undesirable actions.



\subsection{Misconfigure service port in Kubernetes}
A port misconfiguration typically arises from an incorrect port number, which can usually be corrected through patching or updating resource definitions.Importantly, these faults do not inherently require the termination or redeployment of the underlying workloads.

In the context of port misconfiguration, we chose to forbid commands delete and rollout. \texttt{kubectl delete pod}  destroys resources at a fundamental level, removing the pod from the cluster. While this forces Kubernetes to reschedule a new pod, it's an unnecessarily destructive command, and resolve the fault by masking the underlying misconfiguration by replacing it with a clean state. Similarly, the \texttt{kubectl rollout restart}  command, which forces a restart of a deployment, was forbidden because it acts at too broad a level. Together, these restrictions help ensure that the mitigation logic remains focused on the specific, localized aspects of the fault. 

In this case study, we ran the guardrail on the problem  \texttt{misconfig\_app\_hotel\_res-mitigation-1} ten times to assess trigger rate, success rate, and perform analysis of trigger. 

\textbf{General statistics:}
Trigger rate@10: 6/10
Triggered: 512 (delete pod), 638 (delete pod and rollout), 696 (rollout), 448 (rollout), 571 (delete), 014 (delete, rollout)
Not triggered: 794, 178, 787, 045

\textbf{Success rate:} 5/10
(modify format later)
(some examples? but triggers are kinda random, usually after describe)

In the port misconfiguration case study, the safety guardrail was triggered 60\% across 10 trails. The trials where triggers occurred frequently involved the agent attempting to execute both destructive commands  \texttt{delete pod} and \texttt{rollout restart}, which were correctly identified and blocked by the system’s guardrails.
analysis of trigger - immutable resource or destructive command. what happened in each case, and 

\subsection{case study 2}
\subsection{case study 2}