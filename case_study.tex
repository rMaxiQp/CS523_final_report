\section{Case Studies}

In this section, we prove the safety guardrail's effectiveness in practice. Due to the limited funding, we sampled two case studies to showcase our success. For each case, we will report general statistics, safety guardrail trigger rates, agent success rates, followed by a qualitative analysis of observed agent behaviors, blocked commands, and the effectiveness of the system’s guardrails in preventing undesirable actions.

% Out of the six virtualization layer fault, we chose a couple fo test the effectiveness of the safety guardrail. Due to the scope of the project and our limited funding, we outline a list of cases studies that conducted with our implementation. For each case, we report general statistics, trigger rates, and success rates, followed by a qualitative analysis of observed agent behaviors, blocked commands, and the effectiveness of the system’s guardrails in preventing undesirable actions.


\subsection{Misconfigure service port in Kubernetes}

A port misconfiguration araises from an incorrect port assignment. It is a common error, yet leading to broken communication between services, damaging service availability. As the fix, the operator can use patch the node with the correct port number. Importantly, these faults do not inherently require the termination or redeployment of the underlying workloads.

% A port misconfiguration typically arises from an incorrect port number, which can usually be corrected through patching or updating resource definitions. Importantly, these faults do not inherently require the termination or redeployment of the underlying workloads.

In the context of port misconfiguration, we chose to forbid commands delete and rollout. \texttt{kubectl delete pod}  destroys resources at a fundamental level, removing the pod from the cluster. While this forces Kubernetes to reschedule a new pod, it's an unnecessarily destructive command, and resolve the fault by masking the underlying misconfiguration by replacing it with a clean state. Similarly, the \texttt{kubectl rollout restart}  command, which forces a restart of a deployment, was forbidden because it acts at too broad a level. Together, these restrictions help ensure that the mitigation logic remains focused on the specific, localized aspects of the fault. 

In this case study, we ran the guardrail on the problem  \texttt{misconfig\_app\_hotel\_res-mitigation-1} ten times to assess trigger rate, success rate, and perform analysis of trigger. 

\textbf{Trigger rate@10:} 6/10 

\textbf{Success rate:} 5/10

In the port misconfiguration case study, the safety guardrail was triggered 60\% across 10 trails. The trials where triggers occurred frequently involved the agent attempting to execute both destructive commands  \texttt{delete pod} and \texttt{rollout restart}, which were correctly identified and blocked by the system’s guardrails.


\subsection{Assign to non-existent or NotReady node}
In this problem, the fault is injected by assigning the workload’s nodeSelector to a node that does not exist  in the Kubernetes cluster. As a result, pods remain perpetually in the Pending state because the Kubernetes scheduler cannot find any matching node.

Correct mitigation requires removing the invalid nodeSelector to allow the scheduler to target eligible nodes. Importantly, this correction must occur at the Deployment template level. In practice, we noticed that Kubernetes treats a pod’s nodeSelector field as immutable once instantiated; therefore, any attempt to patch the field directly at the pod level will fail with an immutability error. 

The correct mitigation behavior that performs the minimum necessary change is to patch at the deployment level by removing the invalid nodeSelector from the Deployment’s pod template. Kubernetes automatically detects this change and triggers a rollout of new pods that follow the updated template. In the context of this problem, we blocked the commands \texttt{delete} and \texttt{label}. Similar to case 1, \texttt{delete} is an overly destructive operation that was not necessary to mitigate the fault. In addition, it does not address the problematic underlying deployment configuration but merely removes the affected pod. The \texttt{label} command modifies labels on Kubernetes resources. In the context of this problem, the pod is unschedulable because its nodeSelector references a node name that doesn’t exist. To mitigate this, an agent might attempt to relabel an existing node with the missing node’s name. Rather than fixing the Deployment spec, this approach changes the cluster’s infrastructure to “match” the bad configuration and could introduce unintended side effects across the cluster.

In this case study, we ran the guardtail on the problem \texttt{assign\_to\_non\_existent\_node\_social\_net-mitigation-1}

\textbf{Trigger rate@10:} 4/10
%Triggered: 287 (delete pod,success =false), 038(delete pod), 158 (delete, success =false), 602 (delete,success =false)
%Not triggered: 539, 658, 163, 131, 685, 274

\textbf{Success rate:} 7/10

By studying the four logs that triggered the guardrail, we observed that in each case the agent attempted to invoke a \texttt{kubectl delete} command. In three of these cases, the deletion command was accompanied by a failure to successfully mitigate the fault, as the agent failed to identify an alternative, non-destructive resolution path after the command was rejected. In contrast, the six logs that did not trigger the guardrail either avoided issuing any deletion commands or employed alternative commands (e.g., \texttt{kubectl patch deployment}) that directly addressed the faulty \texttt{nodeSelector} field in the Deployment specification. These findings suggest that while the guardrail effectively prevented destructive operations, its presence also challenged the agent to discover less aggressive mitigation strategies. In instances where the agent failed post-trigger, it lacked sufficient logic or number of steps to pivot to a declarative fix. Therefore, the guardrail both improved operational safety and exposed limitations in the agent's fallback behavior following prohibited actions.